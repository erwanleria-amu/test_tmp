\subsubsection*{Master 1 Informatique -\/ Génie Logiciel -\/ Projet Phase 2}

\subsubsection*{Équipe\+: Erwan, Rémy, Yoann, Younes}


\begin{DoxyItemize}
\item Past Sprint\+: 3
\item Next Sprint\+: 4
\item Durée de Sprint\+: 2 semaine
\end{DoxyItemize}

\tabulinesep=1mm
\begin{longtabu} spread 0pt [c]{*{2}{|X[-1]}|}
\hline
\rowcolor{\tableheadbgcolor}\textbf{ Participants }&\textbf{ }\\\cline{1-2}
\endfirsthead
\hline
\endfoot
\hline
\rowcolor{\tableheadbgcolor}\textbf{ Participants }&\textbf{ }\\\cline{1-2}
\endhead
Développeurs &Erwan, Rémy, Yoann, Younes \\\cline{1-2}
Scrum Master &Yoann \\\cline{1-2}
\end{longtabu}
\subsection*{Définition of done}

Dans notre équipe, et donc dans ce document, nous définissons une tâche faite comme une tâche développée dans son entièreté, testée, et commentée. Nous excluons de ce processus la documentation, que nous réaliserons en fin de projet.

\subsection*{Définition du point de charge}

Dans notre équipe, et donc dans ce document, nous définissons un point de charge comme une heure de travail par un développeur. Nous avons évalué la productivité de notre équipe à quatre points de charge par développeur en une semaine, soit au total douze heures par semaines.

\subsection*{Past Sprint Backlog}

\tabulinesep=1mm
\begin{longtabu} spread 0pt [c]{*{3}{|X[-1]}|}
\hline
\rowcolor{\tableheadbgcolor}\textbf{ Item }&\textbf{ Charge }&\textbf{ Fait  }\\\cline{1-3}
\endfirsthead
\hline
\endfoot
\hline
\rowcolor{\tableheadbgcolor}\textbf{ Item }&\textbf{ Charge }&\textbf{ Fait  }\\\cline{1-3}
\endhead
Limitation des villes renvoyées pour éviter une surcharge de l\textquotesingle{}A\+PI &3 &100\% \\\cline{1-3}
Profils &6 &100\% \\\cline{1-3}
Recherches pour mise en place d\textquotesingle{}un chat en temps réel &3 &75\% \\\cline{1-3}
\end{longtabu}
\subsection*{Next Sprint Backlog}

\tabulinesep=1mm
\begin{longtabu} spread 0pt [c]{*{3}{|X[-1]}|}
\hline
\rowcolor{\tableheadbgcolor}\textbf{ Item }&\textbf{ Charge }&\textbf{ Par qui?  }\\\cline{1-3}
\endfirsthead
\hline
\endfoot
\hline
\rowcolor{\tableheadbgcolor}\textbf{ Item }&\textbf{ Charge }&\textbf{ Par qui?  }\\\cline{1-3}
\endhead
Tests fonctionnels divers &6 &Yoann \\\cline{1-3}
Itinéraires &12 &Rémy et Younes \\\cline{1-3}
Documentation &6 &Erwan \\\cline{1-3}
\end{longtabu}
\subsection*{Product Backlog}

\tabulinesep=1mm
\begin{longtabu} spread 0pt [c]{*{4}{|X[-1]}|}
\hline
\rowcolor{\tableheadbgcolor}\textbf{ Item }&\textbf{ Valeur }&\textbf{ Charge }&\textbf{ Risque  }\\\cline{1-4}
\endfirsthead
\hline
\endfoot
\hline
\rowcolor{\tableheadbgcolor}\textbf{ Item }&\textbf{ Valeur }&\textbf{ Charge }&\textbf{ Risque  }\\\cline{1-4}
\endhead
Affichage de la carte &4 &4 &1 \\\cline{1-4}
Météo dans un point donné &4 &3 &1 \\\cline{1-4}
Détection des lieux proches &4 &10 &2 \\\cline{1-4}
Météo d\textquotesingle{}une zone/région &4 &3 &1 \\\cline{1-4}
Documentation, tutoriels... &4 &20 &1 \\\cline{1-4}
Inscription &3 &3 &1 \\\cline{1-4}
Connexion &3 &3 &1 \\\cline{1-4}
Profil &3 &3 &1 \\\cline{1-4}
Système d\textquotesingle{}événements &3 &20 &2 \\\cline{1-4}
Système d\textquotesingle{}amis et de communication &2 &20 &2 \\\cline{1-4}
Lieux favoris &2 &10 &2 \\\cline{1-4}
Gestion d\textquotesingle{}itinéraires &2 &25 &3 \\\cline{1-4}
Intégration des chemins de Grande Randonnée à la carte &2 &Inconnue(Évaluation), 2(Réelle) &3-\/4(Évaluation), 1(Réel) \\\cline{1-4}
Changer la carte &2 &2 &1 \\\cline{1-4}
R\+E\+A\+D\+ME Développeur &2 &3 &1 \\\cline{1-4}
Design du site &1 &20 &1 \\\cline{1-4}
\end{longtabu}
